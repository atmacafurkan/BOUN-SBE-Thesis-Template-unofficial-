\section{Linguistic examples}

In linguistics, we present examples from the language in question for reflecting the structure in the sentence. Since a reader is not necessarily proficient in that language, we provide glossaries for making the sentence open to analysis, and less mysterious. One important point of glossing a linguistic example is aligning the words with the glosses. Luckily for us {\LaTeX} hosts a package called `gb4e' that provides a way to automatize the process. We also use the package `leipzig' for consistent glossing in our examples, and also reporting what glossaries mean in the list of `GLOSSES'. For an example of a linguistic example see (\ref{linguisticexample}).

\begin{exe}
    \ex \label{linguisticexample}
    \gll 
    Furkan ev-i bul-du. \\ F[{\Nom}] house-{\Acc} find-{\Pst}[{\Third}.{\Sg}] \\
    \glt `Furkan found the house.'
\end{exe}

The glosses you use in your examples are automatically added to the list of glosses in the frontmatter of your thesis, including the glosses you can define in `glossaries.tex' like {\Aor}.

\subsection{Phonological Rules}

We write phonological rules for certain changes or ways that a phonological process takes place. To represent these rules we can use `phonrule' package. For example to represent k-zero alternation in Turkish we can have the rule provided in (\ref{finaldevoicing}).

\begin{exe}
\ex \label{finaldevoicing} \phonb{\phonfeat{stop \\ back} }{[\textgamma]}{V}{V}
\end{exe}