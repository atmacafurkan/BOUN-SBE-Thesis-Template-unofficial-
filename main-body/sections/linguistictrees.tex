\section{Linguistic trees}

In linguistics we use `trees' that are representational figures that show hierarchical order of how different levels of structural categories are put together. For the most part of the trees you can use the package `forest'. For example see Figure \ref{fig:linguisticexample} for a representation of (\ref{linguisticexample}).

\begin{figure}[hbt!]
    \centering
    \begin{forest}
        [TP 
            [DP\\\textit{Furkan}$_i$, name=specT]
            [T' 
                [VoiceP 
                    [\sout{DP}$_i$, name=specVoi]
                    [Voice' 
                        [VP 
                            [DP\\\textit{ev-i}]
                            [V\\\textit{bul}]]
                        [Voice]]]
                [T\\\textit{-du}]]]
    \draw[semithick, dashed, ->] (specVoi) to[out=north west, in=south] (specT);
    \end{forest}
    \caption{An example syntax tree}
    \label{fig:linguisticexample}
\end{figure}

We can also employ feature matrices with our trees as in Figure \ref{fig:linguisticexample2}.
      
\begin{figure}[hbt!]
    \centering
    \begin{forest}
        [TP 
            [\begin{avm}
            \[
                per & {\Third} \\ 
                num & {\Sg} \\
                case & {\Nom} \]
            \end{avm}, name=specT]
            [T' 
                [VoiceP 
                    [\sout{DP}$_i$, name=specVoi]
                    [Voice' 
                        [VP 
                            [DP\\\textit{ev-i}]
                            [V\\\textit{bul}]]
                        [Voice]]]
                [T\\\textit{-du}]]]
    \draw[semithick, dashed, ->] (specVoi) to[out=west, in=south] (specT);
    \end{forest}
    \caption{Another example syntax tree}
    \label{fig:linguisticexample2}
\end{figure}

If you happen to have any difficulties with drawing trees consult package documentations for `tikz', `tikz-qtree', `forest', and `avm'. This template lets you use all of them in your trees.

